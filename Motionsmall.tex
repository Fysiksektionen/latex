%%%%%%%%%%%%%%%%%%%%%%%%%%%%%%%%%%%%%%%%%%%%%%%%%%%%%%%%%%%%%%%%%%%%%%%%%%%%%%%%%%%%
%%                                                                                %%
%%    Fysiksektionens mall för motioner.                                          %%
%%                                                                                %%
%%    Skriven av Thomas Lycken (F08) 2011                                         %%
%%    Senast ändrad 2010-02-18 av Thomas Lycken (F-08) tlycken@f.kth.se           %%
%%    Senast ändrad 2011-08-23 av Thomas Lycken (F-08) tlycken@f.kth.se           %%
%%    Senast ändrad 2011-08-24 av Emil Ringh (F-08) eringh@f.kth.se               %%
%%    Senast ändrad 2017-03-06 av Johan Engvall (F-07) jengvall@f.kth.se          %%
%%    Anpassad till fstildok 2018-06-17 av Gustav Gybäck (F15) gyback@kth.se      %%
%%                                                                                %%
%%    Fyll i kommandona nedan för att få en motion                                %%
%%    enligt sektionens grafiska profil.                                          %%
%%                                                                                %%
%%%%%%%%%%%%%%%%%%%%%%%%%%%%%%%%%%%%%%%%%%%%%%%%%%%%%%%%%%%%%%%%%%%%%%%%%%%%%%%%%%%%
% !TEX TS-program = XeLaTeX
% !TEX encoding = utf8
\documentclass{fstildok}

%%%%%%%%%%%%%%%%%%%%%%%%%%%%%%%%%%%%%%%%%%%%%%%%%%%%%%%%%%%%%%%%%%
%%    Ändra följande parametrar, så sköter sig resten själv.    %%					
%%%%%%%%%%%%%%%%%%%%%%%%%%%%%%%%%%%%%%%%%%%%%%%%%%%%%%%%%%%%%%%%%%

% Motion eller Proposition
\typ{Motion}

% Vilket SM det är
\smnr{2}

% Vilket datum SM äger rum
\smdate{2017-03-06}

% Om det är genom en grupp eller nämn fylls det i här. Lämna tom om det inte är det.
\group{\fkm}

% Namn på den eller de som framför motionen
\name{Åsa \\  \F-89 \\ ruckilina@f.kth.se \and A. Nonym \\  F-32 \F örst \\ minmail@kth.se}

% Rubriken till dokumentet. Det kommer stå med i handlingarna.		
\rubrik{En \LaTeX mall för motioner}

% Bakgrunden till motionen. Det kommer som läpande text ovanför ens yrkande.
\bakgrund{					
Fysiksektionen har nu en mall för att skriva snygga motioner. Man kan skriva saker som \F\ är fint, \fkm\ har kul eller $2+2\neq \infty$. Även \noll an kan man skriva om.
}

% Fyll med en attsats för varje förändring du vill införa.
% Du kan här referera till andra attsatser genom \att{nr} som nedan.
\attsatser{ 
\attsats mallen är bra att använda
\attsats bifalla \att{1}
}

%-----------------------------------------------------------------%

\motion